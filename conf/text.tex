\documentclass[18pt]{article}

\usepackage[utf8]{inputenc}
\usepackage[T1,T2A]{fontenc}
\usepackage[english,russian]{babel}
\usepackage{geometry}

\title{Текст доклада}

\begin{document}

% Titular [0.25 min]
\section{Титульный слайд}
\par
Здравствуйте, меня зовут Ельчинов Егор, я учусь на 2 курсе 
бакалавриата ФПМИ МФТИ.

Мой доклад посвящен оптимизации алгоритмов решения 
систем дифференциальных уравнений, описывающих распространение акустической
волны в двумерной среде.

Работа была выполнена в соавторстве с Кудринским Алексеем, студентом 2 курса
бакалавриата ФПМИ МФТИ под научным руководством Хохлова Николая Игоревича, 
кандидата физико~-- математических наук, заместителя заведующего 
лабораторией прикладной вычислительной геофизики МФТИ.

% Introductory [0.5 min]
\section{Вводная часть}
\par
Явные алгоритмы решения дифференциальных уравнений на дискретной 
пространственно~-- временной сетке~--- один из основных методов
аппроксимации волновых процессов.

Наивное решение данной задачи~--- последовательная обработка 
данных внутри вложенных циклов, итерирующихся по всей расчетной сетке.

Характерные параметры сетки обычно достаточно велики, 
поэтому возникает задача эффективного использования 
ресурсов и возможностей современных процессоров, 
таких как многопоточные вычисления, многоуровневое кэширование и 
векторные инструкции.

% Goals & objectives [0.75 min]
\section{Цели и задачи}
\par
Целью работы является эффективная реализация алгоритма 
с использованием возможностей современных процессоров и 
средств современного стандарта языка C++.
Вместо кодогенерации и макроподстановок используются средства вычислений
времени компиляции языка C++, такие как шаблоны и constexpr~-- выражения.

% Tiling algorithms [1 min]
\section{Тайлинг}
\par
Чтобы использовать ресурсы кэша оптимально, необходима временн\'{а}я и 
пространственная локальность обработки данных.
Для этого используется тайлинг~-- разбиение сетки на локальные блоки.

Алгоритм тайлинга ConeFold обеспечивает локальную рекурсивность разбиения
пространства на параллелелограммы с квадратными основаниями, образующими сетку
декартовой системы координат. 
С использованием вычислений времени компиляции современного C++ и 
последующих оптимизаций компилятора, реализация тайлинга типа ConeFold 
позволяет избавиться от промахов кэша инструкций за счет уменьшения количества
условных переходов.

Зависимость времени обработки миллиарда вершин от размера сетки для тайлинга
типа ConeFold показана на графике на слайде.

% Vectorization [1 min]
\section{Векторизация}
\par
Векторизация вычислений позволяет обрабатывать несколько узлов сетки 
одновременно. Для этого были применены AVX~-- инструкции, позволяющие 
производить операции над векторами из четырех чисел двойной точности.

В ходе работы были реализованы два способа локальной векторизации вычислений.
В первом вершины исходной сетки объединялись в квадраты по 4, 
во втором~--- в прямоугольники вдоль оси абсцисс.
Цифрами на рисунках обозначен порядок хранения данных в AVX~-- регистрах.

На графике на слайде показано время работы векторизованного алгоритма в 
сравнении с невекторизованным. Вне зависимости от размера сетки, время работы 
векторизованного алгоритма уменьшается вдвое.

% Multithread computations [1 min]
\section{Многопоточность}
\par
Две многопоточные версии алгоритма были реализованы средствами языка C++ и 
средствами стандарта OpenMP. Данные реализации схожи по эффективности, но в 
сравнении с однопоточной версией, работают в разы медленнее на малом числе 
потоков, что можно видеть на графике зависимости времени работы от 
размера сетки для OpenMP~-- реализации с различным количеством потоков 
в сравнении с однопоточной.

% Data order [1 min]
\section{Хранение данных}
\par
Порядок хранения вершин внутри сетки также оказывает существенное влияние на 
эфективность алгоритма. На слайде представлена зависимость времени 
работы алгоритма от размера сетки c разными способами хранения вершин - 
обычным или линейным порядком, как в наивном алгоритме и Z~-- порядком.

Видно, что использование Z~-- порядка оправдано для больших сеток, так как 
высокая локальность данных становится важнее сложности расчетов координат, 
если большая часть данных лежит вне кэш~-- памяти.

% Conclusion frame [0.5 min]
\section{Заключение}
\par
В качестве результата получены характеристики эффективности различных 
методов параллельных вычислений, векторизации и тайлинга.

Впервые для эффективной реализации алгоритма в данной задаче
вместо кодогенерации и макроподстановок используются средства вычислений
времени компиляции языка C++ современного стандарта C++17.

В дальнейшем планируется увеличить эффективность многопоточных вычислений, 
исследовать задачу на трехмерной сетке, распараллелить алгоритм на CUDA 
и применить описанные методы к моделированию задачи сейсмики 
в упругой постановке и других волновых процессов.

Спасибо за внимание.

% Q&A section

\end{document}

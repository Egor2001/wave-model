% FORMATTING REQUIREMENTS:
% Titular frame
% Introductory frame
% Goals and objectives frame
% Methods employed frame
% Main part frames
% Conclusion frame
% [Final frame]
% 
% FONT REQUIREMENTS:
% Serif fonts (Georgia, Palatino, Times New Roman)
% Size title - [24pt ... 54pt] text - [18pt ... 36pt]
% Bold, Italic - semantic highlights
% No more than 3 font types

\documentclass[presentation,18pt]{beamer}

\usepackage[utf8]{inputenc}
\usepackage[T1,T2A]{fontenc}
\usepackage[english,russian]{babel}
\usepackage{amsmath,amsfonts,amssymb}
\usepackage{geometry}
\geometry{textwidth=24pt}

% Set serif font
\renewcommand{\familydefault}{\sfdefault}

\usetheme{Madrid}
\usecolortheme{whale}

\begin{document}

\title[Оптимизация LRnLA]{Оптимизация решения систем линейных уравнений 
	на двумерной сетке с использованием вных численных схем}
\subtitle{LXIII конференция МФТИ}
\author[Ельчинов]{Ельчинов Е. С.\inst{1} \and
									% Хохлов Н. И. \inst{1} \and
									Кудринский А. М.\inst{2}}
\institute[МФТИ]{МФТИ (ГУ) \\
	\vspace{1ex}
	% TODO: как и куда записывать, coавтор / научрук?
	Научный руководитель~--- Хохлов Н. И., к.ф.-м.н., с.н.с., зам. зав. лаб. \\
	(Лаборатория прикладной вычислительной геофизики МФТИ)
	\vspace{1ex}
}
\date{\today}

% Titular frame
\begin{frame}
	\label{titular}
	\titlepage
\end{frame}

% Introductory frame
\begin{frame}[t]
	\label{introductory}
	\frametitle{Постановка проблемы и актуальность}

	\begin{block}{Проблема}
		Оптимизация моделирования волнового процесса на двумерной сетке
	\end{block}

	\begin{columns}

	\column{0.5\textwidth}
		\begin{alertblock}{}
			\vspace{120pt}
		\end{alertblock}

	\column{0.4\textwidth}
		\begin{block}{Характерные параметры}
			% TODO: уточнить параметры!!!
			\begin{itemize}
				\vspace{1ex}
				\item{$\sim 1\,000\,000\,000$ \\ узлов сетки}

				\vspace{1ex}
				\item{$\sim 1\,000\,000$ \\ временных слоев}

				\vspace{1ex}
				\item{$\sim 1\,000$ \\ задач моделирования}
			\end{itemize}
		\end{block}

	\end{columns}
\end{frame}

% Goals and objectives frame
\begin{frame}[t]
	\label{goals}
	\frametitle{Цели и задачи}

	\begin{block}{Цель}
		Поиск и сравнение эффективности алгоритмов решения задачи
	\end{block}

	\begin{columns}

	\column{0.3\textwidth}
		\begin{block}{Конфигурация алгоритма}
		\end{block}

		\begin{alertblock}{Начальные условия}
			\vspace{80pt}
		\end{alertblock}

	\column{0.2\textwidth}

		\begin{block}{Программный пакет}
			\vspace{80pt}
		\end{block}

	\column{0.3\textwidth}

		\begin{block}{Характеристики эффективности}
		\end{block}

		\begin{alertblock}{Решение задачи}
			\vspace{80pt}
		\end{alertblock}

	\end{columns}
\end{frame}

% Methods employed frame
\begin{frame}[t]
	\label{methods-local}
	\frametitle{Используемые методы}
	\framesubtitle{Локализация данных}

	\begin{columns}

	\column{0.4\textwidth}
		\begin{alertblock}{Порядок вершин}
			\vspace{120pt}
		\end{alertblock}

		\begin{itemize}
			\item Линейный порядок
			\item Z~-- порядок
		\end{itemize}

	\column{0.1\textwidth}

	\column{0.4\textwidth}
		\begin{alertblock}{Тайлинг}
			\vspace{120pt}
		\end{alertblock}

		\begin{itemize}
			\item ConeFold
			\item DiamondTorre
		\end{itemize}

	\end{columns}
\end{frame}

% Methods employed frame
\begin{frame}[t]
	\label{methods-parallel}
	\frametitle{Используемые методы}
	\framesubtitle{Параллельные вычисления}

	\begin{columns}

	\column{0.4\textwidth}
		\begin{alertblock}{Векторизация}
			\vspace{120pt}
		\end{alertblock}

		\begin{itemize}
			\item AVX, шаблон $2 \times 2$
			\item AVX, шаблон $1 \times 4$
		\end{itemize}

	\column{0.1\textwidth}

	\column{0.4\textwidth}
		\begin{alertblock}{Многопоточность}
			\vspace{120pt}
		\end{alertblock}

		\begin{itemize}
			\item std::thread (C++17)
			\item стандарт OpenMP
		\end{itemize}

	\end{columns}
\end{frame}

% Main part frames
\begin{frame}[t]
	\label{data-order}
	\frametitle{Способы хранения данных}

	\begin{columns}

	\column{0.4\textwidth}
		\begin{alertblock}{Линейный порядок}
			\vspace{120pt}
		\end{alertblock}

		\begin{itemize}
			\item Простые расчеты
			\item Невысокая локальность
		\end{itemize}

	\column{0.1\textwidth}

	\column{0.4\textwidth}
		\begin{alertblock}{Z~-- порядок}
			\vspace{120pt}
		\end{alertblock}

		\begin{itemize}
			\item Более сложные расчеты
			\item Высокая локальность
		\end{itemize}

	\end{columns}
\end{frame}

\begin{frame}[t]
	\label{tiling}
	\frametitle{Алгоритмы тайлинга}

	\begin{columns}

	\column{0.4\textwidth}
		\begin{alertblock}{ConeFold}
			\vspace{120pt}
		\end{alertblock}

		\begin{itemize}
			\item Локально~-- рекурсивный
			\item Средняя асинхронность
		\end{itemize}

	\column{0.1\textwidth}

	\column{0.4\textwidth}
		\begin{alertblock}{DiamondTorre}
			\vspace{120pt}
		\end{alertblock}

		\begin{itemize}
			\item Нерекурсивный
			\item Высокая асинхронность
		\end{itemize}

	\end{columns}
\end{frame}

\begin{frame}
	\label{vectorization}
	\frametitle{Виды локальных шаблонов и векторизация}
\end{frame}

\begin{frame}
	\label{multithread}
	\frametitle{Многопоточная обработка}
\end{frame}

\begin{frame}
	\frametitle{Результаты}
\end{frame}

% Conclusion frame
\begin{frame}
	\label{conclusion}
	\frametitle{Заключение}
\end{frame}

% Materials frame
\begin{frame}
	\label{materials}
	\frametitle{Материалы}
\end{frame}

% [Final frame]
% \begin{frame}
% 	\frametitle{Спасибо за внимание}
% \end{frame}

\end{document}
